% Options for packages loaded elsewhere
\PassOptionsToPackage{unicode}{hyperref}
\PassOptionsToPackage{hyphens}{url}
%
\documentclass[
]{article}
\usepackage{amsmath,amssymb}
\usepackage{lmodern}
\usepackage{iftex}
\ifPDFTeX
  \usepackage[T1]{fontenc}
  \usepackage[utf8]{inputenc}
  \usepackage{textcomp} % provide euro and other symbols
\else % if luatex or xetex
  \usepackage{unicode-math}
  \defaultfontfeatures{Scale=MatchLowercase}
  \defaultfontfeatures[\rmfamily]{Ligatures=TeX,Scale=1}
\fi
% Use upquote if available, for straight quotes in verbatim environments
\IfFileExists{upquote.sty}{\usepackage{upquote}}{}
\IfFileExists{microtype.sty}{% use microtype if available
  \usepackage[]{microtype}
  \UseMicrotypeSet[protrusion]{basicmath} % disable protrusion for tt fonts
}{}
\makeatletter
\@ifundefined{KOMAClassName}{% if non-KOMA class
  \IfFileExists{parskip.sty}{%
    \usepackage{parskip}
  }{% else
    \setlength{\parindent}{0pt}
    \setlength{\parskip}{6pt plus 2pt minus 1pt}}
}{% if KOMA class
  \KOMAoptions{parskip=half}}
\makeatother
\usepackage{xcolor}
\usepackage[margin=1in]{geometry}
\usepackage{color}
\usepackage{fancyvrb}
\newcommand{\VerbBar}{|}
\newcommand{\VERB}{\Verb[commandchars=\\\{\}]}
\DefineVerbatimEnvironment{Highlighting}{Verbatim}{commandchars=\\\{\}}
% Add ',fontsize=\small' for more characters per line
\usepackage{framed}
\definecolor{shadecolor}{RGB}{48,48,48}
\newenvironment{Shaded}{\begin{snugshade}}{\end{snugshade}}
\newcommand{\AlertTok}[1]{\textcolor[rgb]{1.00,0.81,0.69}{#1}}
\newcommand{\AnnotationTok}[1]{\textcolor[rgb]{0.50,0.62,0.50}{\textbf{#1}}}
\newcommand{\AttributeTok}[1]{\textcolor[rgb]{0.80,0.80,0.80}{#1}}
\newcommand{\BaseNTok}[1]{\textcolor[rgb]{0.86,0.64,0.64}{#1}}
\newcommand{\BuiltInTok}[1]{\textcolor[rgb]{0.80,0.80,0.80}{#1}}
\newcommand{\CharTok}[1]{\textcolor[rgb]{0.86,0.64,0.64}{#1}}
\newcommand{\CommentTok}[1]{\textcolor[rgb]{0.50,0.62,0.50}{#1}}
\newcommand{\CommentVarTok}[1]{\textcolor[rgb]{0.50,0.62,0.50}{\textbf{#1}}}
\newcommand{\ConstantTok}[1]{\textcolor[rgb]{0.86,0.64,0.64}{\textbf{#1}}}
\newcommand{\ControlFlowTok}[1]{\textcolor[rgb]{0.94,0.87,0.69}{#1}}
\newcommand{\DataTypeTok}[1]{\textcolor[rgb]{0.87,0.87,0.75}{#1}}
\newcommand{\DecValTok}[1]{\textcolor[rgb]{0.86,0.86,0.80}{#1}}
\newcommand{\DocumentationTok}[1]{\textcolor[rgb]{0.50,0.62,0.50}{#1}}
\newcommand{\ErrorTok}[1]{\textcolor[rgb]{0.76,0.75,0.62}{#1}}
\newcommand{\ExtensionTok}[1]{\textcolor[rgb]{0.80,0.80,0.80}{#1}}
\newcommand{\FloatTok}[1]{\textcolor[rgb]{0.75,0.75,0.82}{#1}}
\newcommand{\FunctionTok}[1]{\textcolor[rgb]{0.94,0.94,0.56}{#1}}
\newcommand{\ImportTok}[1]{\textcolor[rgb]{0.80,0.80,0.80}{#1}}
\newcommand{\InformationTok}[1]{\textcolor[rgb]{0.50,0.62,0.50}{\textbf{#1}}}
\newcommand{\KeywordTok}[1]{\textcolor[rgb]{0.94,0.87,0.69}{#1}}
\newcommand{\NormalTok}[1]{\textcolor[rgb]{0.80,0.80,0.80}{#1}}
\newcommand{\OperatorTok}[1]{\textcolor[rgb]{0.94,0.94,0.82}{#1}}
\newcommand{\OtherTok}[1]{\textcolor[rgb]{0.94,0.94,0.56}{#1}}
\newcommand{\PreprocessorTok}[1]{\textcolor[rgb]{1.00,0.81,0.69}{\textbf{#1}}}
\newcommand{\RegionMarkerTok}[1]{\textcolor[rgb]{0.80,0.80,0.80}{#1}}
\newcommand{\SpecialCharTok}[1]{\textcolor[rgb]{0.86,0.64,0.64}{#1}}
\newcommand{\SpecialStringTok}[1]{\textcolor[rgb]{0.80,0.58,0.58}{#1}}
\newcommand{\StringTok}[1]{\textcolor[rgb]{0.80,0.58,0.58}{#1}}
\newcommand{\VariableTok}[1]{\textcolor[rgb]{0.80,0.80,0.80}{#1}}
\newcommand{\VerbatimStringTok}[1]{\textcolor[rgb]{0.80,0.58,0.58}{#1}}
\newcommand{\WarningTok}[1]{\textcolor[rgb]{0.50,0.62,0.50}{\textbf{#1}}}
\usepackage{graphicx}
\makeatletter
\def\maxwidth{\ifdim\Gin@nat@width>\linewidth\linewidth\else\Gin@nat@width\fi}
\def\maxheight{\ifdim\Gin@nat@height>\textheight\textheight\else\Gin@nat@height\fi}
\makeatother
% Scale images if necessary, so that they will not overflow the page
% margins by default, and it is still possible to overwrite the defaults
% using explicit options in \includegraphics[width, height, ...]{}
\setkeys{Gin}{width=\maxwidth,height=\maxheight,keepaspectratio}
% Set default figure placement to htbp
\makeatletter
\def\fps@figure{htbp}
\makeatother
\setlength{\emergencystretch}{3em} % prevent overfull lines
\providecommand{\tightlist}{%
  \setlength{\itemsep}{0pt}\setlength{\parskip}{0pt}}
\setcounter{secnumdepth}{-\maxdimen} % remove section numbering
\ifLuaTeX
  \usepackage{selnolig}  % disable illegal ligatures
\fi
\IfFileExists{bookmark.sty}{\usepackage{bookmark}}{\usepackage{hyperref}}
\IfFileExists{xurl.sty}{\usepackage{xurl}}{} % add URL line breaks if available
\urlstyle{same} % disable monospaced font for URLs
\hypersetup{
  pdftitle={M2.951 - Tipologia i cicle de vida de les dades},
  pdfauthor={Autors: Francisco J. Bastida López (fbastidal@uoc.edu) / Ivan Benaiges Trenchs (ibenaiges@uoc.edu)},
  hidelinks,
  pdfcreator={LaTeX via pandoc}}

\title{M2.951 - Tipologia i cicle de vida de les dades}
\author{Autors: Francisco J. Bastida López
(\href{mailto:fbastidal@uoc.edu}{\nolinkurl{fbastidal@uoc.edu}}) / Ivan
Benaiges Trenchs
(\href{mailto:ibenaiges@uoc.edu}{\nolinkurl{ibenaiges@uoc.edu}})}
\date{Juny 2023}

\begin{document}
\maketitle

{
\setcounter{tocdepth}{2}
\tableofcontents
}
\begin{center}\rule{0.5\linewidth}{0.5pt}\end{center}

\hypertarget{pruxe0ctica-2-25-nota-final}{%
\section{Pràctica 2 (25\% nota
final)}\label{pruxe0ctica-2-25-nota-final}}

\begin{center}\rule{0.5\linewidth}{0.5pt}\end{center}

\hypertarget{descripciuxf3-de-la-pruxe0ctica-a-realitzar}{%
\subsection{Descripció de la Pràctica a
realitzar}\label{descripciuxf3-de-la-pruxe0ctica-a-realitzar}}

L'objectiu d'aquesta activitat serà el tractament d'un dataset, que pot
ser el creat a la pràctica 1 o bé qualsevol dataset lliure disponible a
Kaggle (\url{https://www.kaggle.com}).

Un exemple de dataset amb el qual podeu treballar és el ``Heart Attack
Analysis \& Prediction dataset''
(\url{https://www.kaggle.com/datasets/rashikrahmanpritom/heart-attack-analysis-predictiondataset}).

Important: si escolliu un dataset diferent al proposat és important que
aquest contingui una àmplia varietat de dades numèriques i categòriques
per poder fer una anàlisi més rica i poder respondre a les diferents
preguntes plantejades a l'enunciat de la pràctica. Seguint les
principals etapes d'un projecte analític, les diferents tasques a
realitzar (i justificar) són les següents:

\begin{enumerate}
\def\labelenumi{\arabic{enumi}.}
\tightlist
\item
  Descripció del dataset. Perquè és important i quina pregunta/problema
  pretén respondre?
\item
  Integració i selecció de les dades d'interès a analitzar. Pot ser el
  resultat d'addicionar diferents datasets o una subselecció útil de les
  dades originals, en base a l'objectiu que es vulgui aconseguir.
\item
  Neteja de les dades. 3.1. Les dades contenen zeros o elements buits?
  Gestiona cadascun d'aquests casos. 3.2. Identifica i gestiona els
  valors extrems.
\item
  Anàlisi de les dades. 4.1. Selecció dels grups de dades que es volen
  analitzar/comparar (p.~e., si es volen comparar grups de dades, quins
  són aquests grups i quins tipus d'anàlisi s'aplicaran?). 4.2.
  Comprovació de la normalitat i homogeneïtat de la variància. 4.3.
  Aplicació de proves estadístiques per comparar els grups de dades. En
  funció de les dades i de l'objectiu de l'estudi, aplicar proves de
  contrast d'hipòtesis, correlacions, regressions, etc. Aplicar almenys
  tres mètodes d'anàlisi diferents.
\item
  Representació dels resultats a partir de taules i gràfiques. Aquest
  apartat es pot respondre al llarg de la pràctica, sense la necessitat
  de concentrar totes les representacions en aquest punt de la pràctica.
\item
  Resolució del problema. A partir dels resultats obtinguts, quines són
  les conclusions? Els resultats permeten respondre al problema?
\item
  Codi. Cal adjuntar el codi, preferiblement en R, amb el que s'ha
  realitzat la neteja, anàlisi i representació de les dades. Si ho
  preferiu, també podeu treballar en Python.
\item
  Vídeo. Realitzar un breu vídeo explicatiu de la pràctica (màxim 10
  minuts) on tots els integrants de l'equip expliquin amb les seves
  pròpies paraules el desenvolupament de la pràctica, basant-se en les
  preguntes de l'enunciat per a justificar i explicar el codi
  desenvolupat. Aquest vídeo s'haurà de lliurar a través d'un enllaç al
  Google Drive de la UOC (\url{https://drive.google.com/}\ldots),
  juntament amb l'enllaç al repositori Git lliurat.
\end{enumerate}

\begin{center}\rule{0.5\linewidth}{0.5pt}\end{center}

\hypertarget{exercici-1-descripciuxf3-del-dataset}{%
\subsection{Exercici 1: Descripció del
dataset}\label{exercici-1-descripciuxf3-del-dataset}}

\hypertarget{exercici-2-integraciuxf3-i-selecciuxf3-de-les-dades-dinteruxe8s-a-analitzar}{%
\subsection{Exercici 2: Integració i selecció de les dades d'interès a
analitzar}\label{exercici-2-integraciuxf3-i-selecciuxf3-de-les-dades-dinteruxe8s-a-analitzar}}

Abans de començar, és necessari carregar el fitxer en un data frame que
ens permeti treballar de forma còmode amb les dades:

\begin{Shaded}
\begin{Highlighting}[]
\CommentTok{\# El joc de dades està conformat per un fitxer CSV, amb separació per comes i amb una capçalera, pel que carreguem el fitxer utilitzant la següent funció:}
\NormalTok{dfHeartAttack }\OtherTok{\textless{}{-}} \FunctionTok{read.csv}\NormalTok{(}\StringTok{"..}\SpecialCharTok{\textbackslash{}\textbackslash{}}\StringTok{dataset}\SpecialCharTok{\textbackslash{}\textbackslash{}}\StringTok{heart.csv"}\NormalTok{, }\AttributeTok{header=}\ConstantTok{TRUE}\NormalTok{)}
\end{Highlighting}
\end{Shaded}

\hypertarget{exercici-3-neteja-de-les-dades}{%
\subsection{Exercici 3: Neteja de les
dades}\label{exercici-3-neteja-de-les-dades}}

\hypertarget{exercici-4-anuxe0lisi-de-les-dades}{%
\subsection{Exercici 4: Anàlisi de les
dades}\label{exercici-4-anuxe0lisi-de-les-dades}}

\hypertarget{exercici-5-representaciuxf3-dels-resultats-a-partir-de-taules-i-gruxe0fiques}{%
\subsection{Exercici 5: Representació dels resultats a partir de taules
i
gràfiques}\label{exercici-5-representaciuxf3-dels-resultats-a-partir-de-taules-i-gruxe0fiques}}

\hypertarget{exercici-6-resoluciuxf3-del-problema}{%
\subsection{Exercici 6: Resolució del
problema}\label{exercici-6-resoluciuxf3-del-problema}}

\end{document}
